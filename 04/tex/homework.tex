\documentclass[a4paper, 11pt]{article}
\usepackage{comment} 
\usepackage{fullpage}
\usepackage{amsmath} 
\usepackage{amssymb} 
\usepackage{mathtools}
\usepackage{siunitx}
\usepackage{xfrac}
\usepackage{icomma}
\usepackage[section,below]{placeins}
\usepackage[labelfont=bf,font=small,width=0.9\textwidth]{caption}
\usepackage{subcaption}
\usepackage{graphicx}
\usepackage{grffile}
\usepackage{float}
\floatplacement{figure}{htbp}
\floatplacement{table}{htbp}
\usepackage{booktabs}
\usepackage{hyperref}
\sisetup{separate-uncertainty=true}

\begin{document}
\noindent
\centerline{\small{\textsc{Michigan State University}}} \\
\large{\textbf{CMSE 823 – Numerical Linear Algebra \hfill Spring 2020 \\
Homework 4}} \\
Alexander Harnisch \\
\noindent\makebox[\linewidth]{\rule{\textwidth}{0.4pt}}

Hey He!

I choose to use Python for the coding assignments. I'm sure you're familiar with how to execute Python scripts. For this assignment everything should be very self explanatory. Just run the \textit{qr.py} script and it should generate the following output:
\begin{verbatim}
Case 1:
-------
Check for correctness:
Classical Gram-Schmidt: ||QR - A|| = 1.4936523181711914e-15
Modified Gram-Scmidt: ||QR - A|| = 1.1801832636420706e-15
Householder Transform based: ||QR - A|| = 2.5414166642833352e-15
Numpy: ||QR - A|| = 2.739800093299453e-15

Check orthogonality
Classical Gram-Schmidt: ||Q*Q - I|| = 4.292074685283655e-15
Modified Gram-Scmidt: ||Q*Q - I|| = 1.0481175698739523e-15
Householder Transform based: ||Q*Q - I|| = 9.940313666921538e-16
Numpy: ||Q*Q - I|| = 3.4625011052432235e-16

Case 2:
-------
Check for correctness:
Classical Gram-Schmidt: ||QR - A|| = 0.0
Modified Gram-Scmidt: ||QR - A|| = 0.0
Householder Transform based: ||QR - A|| = 2.7194799110210365e-16
Numpy: ||QR - A|| = 2.7194799110210365e-16

Check orthogonality
Classical Gram-Schmidt: ||Q*Q - I|| = 3.2547231622285317e-11
Modified Gram-Scmidt: ||Q*Q - I|| = 3.2547231622285317e-11
Householder Transform based: ||Q*Q - I|| = 0.0
Numpy: ||Q*Q - I|| = 2.3551386880256624e-16
\end{verbatim}

There is not a lot to observe except for the very obvious: All methods seem to
work correctly and generate the expected result. Even though they are not
directly comparable for the Householder method, because it is the only one that
constructs a full QR decomposition while the others only construct reduced
decompositions.

As expected we see in both cases that the householder transform method (and
NumPy's built in one, which I believe uses householder most of the time) is
more accurate (the norms are closer to zero by one order of magnitude). 

In the first case the modified Gram-Schmidt implementation seems to be slightly
better than the Classical, but the difference is not significant. However, with
matrices that small and only two cases it is very hard to make any
statistically significant observations.

Maybe one more thing to observe: Since the Gram-Schmidt based implementations
do not touch the lower triangular part of $R$, they stay exactly zero. With the
householder method they are numerical zeros which are some small number very
close to zero due to rounding errors. It would be legitimate to set these
values to exactly zero, since we know from theory that they are indeed zero.
However, I have not implemented that here.

Additional comment: I did not print the actual results here, just looked at
them during debugging in the console. The assignment says ``compute the
results'', not sure if you want us to explicitly state $Q$ and $R$ here, I
don't see the point. Please let me know if that's what you want!

Also, more of a question than a comment: In the book Algorithm 10.1 states the
expression:
\begin{equation}
  v_k(v_k^*A_{k:m,k:n})
\end{equation}
I figured out what it means, but technically this is not a mathematically
correct expression is it? $v_k^*A_{k:m,k:n}$ is a column vector and so is
$v_k$. This is not a valid outer or inner product. The expected result here is
a matrix where $v_k$ is multiplied with each entry of the vector on the right
to produce a matrix with the products in each column, am I right? At least
that's how I have implemented it and it gives the correct result. 

\end{document}
